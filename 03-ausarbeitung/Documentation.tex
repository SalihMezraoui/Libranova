%------------------ vorlage.tex ------------------------------------------------
%
% LaTeX-Vorlage zur Erstellung von Projektdokumentationen
% im Fachbereich Informatik der Hochschule Trier
%
% Basis: Vorlage 'svmono' des Springer Verlags
% Bearbeiter: Hermann Schloß, Christian Bettinger
%
%-------------------------------------------------------------------------------


%------------------ Präambel ---------------------------------------------------
\documentclass[envcountsame, envcountchap, deutsch]{i-studis}

\usepackage[utf8]{inputenc}

\usepackage[a4paper]{geometry}
\usepackage[english, ngerman]{babel}

\usepackage[pdftex]{graphicx}
\usepackage{epstopdf}

\usepackage{listings}
\usepackage{titlesec}
\setcounter{secnumdepth}{4}

\usepackage[german, ruled, vlined]{algorithm2e}
\usepackage{amssymb, amsfonts, amstext, amsmath}
\usepackage{array}
\usepackage[skip=10pt]{caption}
\usepackage[usenames, dvipsnames]{color}
\usepackage[pdftex, plainpages=false]{hyperref}
\usepackage{textcomp}

\usepackage{bibgerm}
\bibliographystyle{geralpha}

\usepackage{xcolor} % Include the xcolor package for color definitions
\definecolor{lightgray}{gray}{0.9} % Define the lightgray color

\usepackage{graphicx}
\usepackage{placeins} % For \FloatBarrier
\usepackage{float} % Include this package for the [H] option


\usepackage{longtable}
\usepackage{caption}
\usepackage{float}

\usepackage{listings}
\lstdefinelanguage{ini}{
	basicstyle=\ttfamily\small,
	sensitive=false,
	morecomment=[s][\color{gray}]{;}{\newline},
	morecomment=[l]{\#},
	morecomment=[l]{//},
	morestring=[b]",
	morekeywords={},
}


\makeatletter
\def\LT@makecaption#1#2#3{%
	\global\setbox\@tempboxa\hbox{\textbf{#1~#2}: #3}%
	\ifdim \wd\@tempboxa >\hsize
	\textbf{#1~#2}: #3\par
	\else
	\hbox to\hsize{\hfil\box\@tempboxa\hfil}%
	\fi
	\vskip\belowcaptionskip
}
\makeatother

\newcounter{frcounter}
\newcommand{\fr}[1]{%
	\stepcounter{frcounter}%
	FR\arabic{frcounter} & #1 \\ \hline
}

\newcounter{nfrcounter}
\newcommand{\nfr}[1]{%
	\stepcounter{nfrcounter}%
	NFR\arabic{nfrcounter} & #1 \\ \hline
}

\newcounter{trcounter}
\newcommand{\tr}[1]{%
	\stepcounter{trcounter}%
	TR\arabic{trcounter} & #1 \\ \hline
}




\usepackage{makeidx}
\usepackage{multicol}
\makeindex

\pagestyle{myheadings}
\setlength{\textheight}{1.1\textheight}

\lstset{
	basicstyle=\scriptsize\ttfamily,
	commentstyle=\scriptsize\ttfamily\color{Gray},
	identifierstyle=\scriptsize\ttfamily,
	keywordstyle=\scriptsize\ttfamily,
	stringstyle=\scriptsize\ttfamily,
	tabsize=4,
	numbers=left,
	numberstyle=\tiny,
	numberblanklines=false,
	frame=single,
	framesep=3mm,
	framexleftmargin=7mm,
	xleftmargin=10mm,
	linewidth=144mm,
	captionpos=b,
}


%------------------ Manuelle Silbentrennung ------------------------------------
\hyphenation{Ele-men-tar-ob-jek-te ab-ge-tas-tet Aus-wer-tung House-holder-Matrix Least-Squares-Al-go-ri-th-men}


%------------------ Titelseite -------------------------------------------------
\begin{document}

\title{„LibraNova“: Full-Stack-Bibliotheksmanagementapp für die Ausleihe von Büchern}
\subtitle{"LibraNova" – A Full-Stack Library Management System for Lending Books}

\author{Bearbeiter: Mohammed Salih Mezraoui}

\supervisor{Prof. Dr. Georg Schneider}

\address{Trier}
\submitdate{den 15.08.2025}

%------------------ Projektart -------------------------------------------------
%\project{Bachelor-Projektarbeit}
\project{Bachelor-Abschlussarbeit}
%\project{Master-Projektstudium}
%\project{Master-Abschlussarbeit}
%\project{Seminar}
%\project{Hausarbeit}

\mytitlepage

%------------------ Vorwort, Kurzfassung, Verzeichnisse ------------------------
\frontmatter
%\input{chapters/Vorwort}								% Vorwort (optional)
\kurzfassung

Die Arbeit stellt „LibraNova“ vor – eine moderne und benutzerfreundliche Full-Stack-Webanwendung für das digitale Bibliotheksmanagement und die Ausleihe von Büchern. Entwickelt mit Spring Boot im Backend und React im Frontend, ermöglicht die Anwendung Nutzer:innen das Suchen von Büchern, das Prüfen ihrer Verfügbarkeit sowie das Lesen und Verfassen von Reviews. Administrator:innen können den Buchbestand verwalten und auf Anfragen der Nutzer:innen reagieren. Für Authentifizierung und Autorisierung kommen moderne Sicherheitsstandards wie Okta mit JWT, OAuth2 und OpenID Connect zum Einsatz. Die Zahlungsabwicklung kostenpflichtiger Dienste erfolgt über Stripe. Eine relationale MySQL-Datenbank gewährleistet eine strukturierte und effiziente Datenhaltung. Besonderer Fokus liegt auf intuitiver Benutzerführung, responsivem Design und automatisierten Tests (JUnit, Mockito) zur Qualitätssicherung.



							% Kurzfassung/Abstract
\tableofcontents										% Inhaltsverzeichnis
\listoffigures											% Abbildungsverzeichnis (optional)
%\listoftables											% Tabellenverzeichnis (optional)
%\lstlistoflistings										% Listings (optional)


%------------------ Kapitel ----------------------------------------------------
\mainmatter
\chapter{Einleitung und Anforderungen}
In diesem Kapitel werden zentrale Aspekte der Anwendung „LibraNova“ vorgestellt. Der Fokus liegt dabei auf:

\begin{itemize}
	\item Einem kompakten Überblick über die Ziele und Funktionen der entwickelten Bibliotheksanwendung.
	\item Sowie der Erhebung und Beschreibung der funktionalen, nicht-funktionalen und technischen Anforderungen, die im Rahmen der Entwicklung berücksichtigt wurden.
\end{itemize}

\section{Überblick}\index{Überblick}

„LibraNova“ ist ein modernes Full-Stack-Bibliotheksmanagementsystem, das eine einfache und effiziente Verwaltung von Buchausleihen ermöglicht. Nutzer können Bücher bequem suchen, ausleihen, bewerten und ihre Ausleihen verfolgen – alles über eine benutzerfreundliche und für mobile Geräte optimierte Webanwendung. Die Plattform funktioniert zuverlässig auf verschiedenen Geräten und Bildschirmgrößen.\\
Neben der übersichtlichen Oberfläche bietet „LibraNova“ wichtige Funktionen wie eine Echtzeit-Anzeige der Buchverfügbarkeit, ein Bewertungssystem sowie eine sichere Nutzeranmeldung mit JWT und OAuth2. Administratoren können Bücher verwalten und auf Nutzeranfragen reagieren. Die Integration externer Dienste wie Stripe ermöglicht sichere Zahlungen für kostenpflichtige Funktionen. Technologisch basiert die Anwendung auf Spring Boot, React und MySQL und ist somit gut skalierbar, sicher und leistungsfähig.

\section{Anforderungen}\index{Anforderungen}

Die folgenden Unterabschnitte beschreiben die zentralen Anforderungen an das System. Dazu zählen funktionale, nicht-funktionale sowie technische Anforderungen, die für eine strukturierte Umsetzung der Anwendung erforderlich sind.

\subsection{Funktionale Anforderungen}

Funktionale Anforderungen definieren, welche Dienste das System leisten soll und wie es sich bei bestimmten Eingaben oder in bestimmten Situationen verhalten soll.\\ \\
Die Tabelle \ref{tab:functional-requirements} zeigt alle funktionalen Anforderungen der Anwendung.

\begin{longtable}{|c|p{13cm}|}
	\hline
	\textbf{ID} & \textbf{Anforderung (Beschreibung)} \\
	\hline
	\endfirsthead
	
	% No header repetition on next pages
	\endhead
	
	% Step 2: Use \fr{...} to add entries
	\fr{Nutzer können Bücher nach Titel oder Kategorie suchen.}
	\fr{Nutzer können die Verfügbarkeit eines Buches in Echtzeit einsehen.}
	\fr{Nutzer können bis zu fünf Bücher gleichzeitig ausleihen.}
	\fr{Nutzer können ihre ausgeliehenen Bücher einsehen.}
	\fr{Nutzer können ihre Ausleihhistorie einsehen.}
	\fr{Nutzer können ein Buch bewerten.}
	\fr{Nutzer können eine Rezension verfassen.}
	\fr{Nutzer können Rezensionen anderer Nutzer lesen.}
	\fr{Das System zeigt eine Übersicht der verfügbaren Bücher im Katalog an.}
	\fr{Administratoren können neue Bücher zum Katalog hinzufügen.}
	\fr{Administratoren können die Anzahl der Exemplare eines vorhandenen Buches erhöhen oder verringern.}
	\fr{Administratoren können Bücher aus dem Katalog löschen.}
	\fr{Nutzer können sich in ihr persönliches Konto einloggen.}
	\fr{Das System verwaltet Nutzerrollen (z.\,B. Nutzer, Administrator).}
	\fr{Nutzer können die Leihfrist eines Buches um bis zu 7 Tage verlängern.}
	\fr{Nutzer können ausgeliehene Bücher zurückgeben.}
	\fr{Das System zeigt an, wenn ein Buch nicht verfügbar ist.}
	\fr{Für kostenpflichtige Dienstleistungen ist eine Zahlungsabwicklung über Stripe integriert.}
	\fr{Nutzer müssen eingeloggt sein, um ein Buch auszuleihen.}
	\fr{Nutzer müssen eingeloggt sein, um eine Rezension verfassen zu können.}
	\fr{Nutzer müssen eingeloggt sein, um ein Buch bewerten zu können.}
	\fr{Das System zeigt an, wie viele Bücher von den maximal fünf gleichzeitig ausgeliehen sind.}
	\fr{Das System zeigt die Anzahl der Suchergebnisse basierend auf eingegebenen Suchbegriffen und gewählten Kategorien an.}
	\fr{Das System zeigt Bücherlisten (Suchergebnisse, ausgeliehene Bücher, Ausleihhistorie) paginiert an.}
	\fr{Das System muss sicherstellen, dass Nutzer sowohl einen Nachrichtentitel als auch den Nachrichtentext angeben, bevor sie eine Anfrage absenden können.}
	\fr{Das System muss sicherstellen, dass alle erforderlichen Felder beim Anlegen eines neuen Buches ausgefüllt sind.}
	\fr{Das System muss sicherstellen, dass Administratoren das Antwortfeld ausfüllen, bevor sie eine Antwort absenden können.}
	\fr{Das System muss im Ausleihverlauf eines Nutzers sowohl das Ausleihdatum als auch das Rückgabedatum jedes Buches anzeigen.}
	
	
	\caption{Funktionale Anforderungen von LibraNova}
	\label{tab:functional-requirements}
\end{longtable}



\subsection{Nich-funktionale Anforderungen}
Nicht-funktionale Anforderungen beschreiben die Qualitätsmerkmale und Randbedingungen des Systems, wie etwa Leistung, Sicherheit, Benutzbarkeit und Zuverlässigkeit\\ \\
Die folgende Tabelle \ref{tab:non-functional-requirements} fasst die wichtigsten nicht-funktionalen Anforderungen für LibraNova zusammen.

\begin{longtable}{|c|p{13cm}|}
	\hline
	\textbf{ID} & \textbf{Anforderung (Beschreibung)} \\
	\hline
	\endfirsthead
	
	\hline
	\textbf{ID} & \textbf{Anforderung (Beschreibung)} \\
	\hline
	\endhead
	
	\nfr{Das System muss eine responsive Benutzeroberfläche bieten, die auf verschiedenen Geräten und Bildschirmgrößen funktioniert.}
	\nfr{Die Anwendung muss eine sichere Kommunikation über HTTPS (TLS) gewährleisten, um die Datenübertragung zwischen Frontend und Backend zu schützen. (Ports 8443 und 3000).}
	\nfr{Die Authentifizierung und Autorisierung erfolgt über Okta als Identity Provider unter Verwendung von OAuth2, OpenID Connect und JWT, um sichere und standardisierte Benutzerzugriffe zu gewährleisten.}
	\nfr{Das System muss Cross-Origin Resource Sharing (CORS) korrekt konfigurieren, um Anfragen nur von der vertrauenswürdigen Frontend-Domain \texttt{https://localhost:3000} zuzulassen.}
	\nfr{Das System muss sicherstellen, dass Benutzer nur mit gültigen und nicht abgelaufenen Tokens Zugriff auf geschützte Ressourcen erhalten.}
	\nfr{Die Reaktionszeit der Such- und Filterfunktionen im Frontend darf 2 Sekunden nicht überschreiten, um ein flüssiges Nutzererlebnis zu gewährleisten.}
	\nfr{Die Anwendung muss grundlegende Barrierefreiheitsanforderungen gemäß WCAG 2.1 (Level AA) erfüllen, um auch Nutzern mit Einschränkungen einen uneingeschränkten Zugang und eine barrierefreie Nutzung zu ermöglichen.}
	\nfr{Die REST-APIs müssen robust gegen fehlerhafte Eingaben sein und validierte Anfragen verarbeiten, um die Stabilität des Systems zu gewährleisten.}
	\nfr{Die Backend-Services sollen modular aufgebaut sein, um einfache Wartbarkeit und Erweiterbarkeit zu ermöglichen.}
	\nfr{Die REST-API-Dokumentation muss aktuell und für Entwickler leicht zugänglich sein (über Swagger UI).}
	\nfr{Die Anwendung soll eine klare Trennung zwischen Benutzerrollen (Nutzer, Administrator) gewährleisten und Zugriffsrechte strikt durchsetzen.}
	\nfr{Die Anwendung muss auf allen gängigen Webbrowsern wie Firefox, Chrome und Safari ohne Einschränkungen lauffähig sein.}
	\nfr{NFR31: Der Anmeldevorgang eines Nutzers (ab dem Klick auf den Login-Button bis zur erfolgreichen Authentifizierung und Weiterleitung zur Startseite) darf nicht länger als 3 Sekunden dauern.}
	\nfr{Die Benutzeroberfläche muss auf allen Seiten konsistente Navigationselemente, Schaltflächen-Designs und Layouts verwenden, um eine einheitliche Nutzererfahrung sicherzustellen.}
	\nfr{Die Benutzerführung muss intuitiv gestaltet sein, sodass typische Nutzeraktionen wie Buchsuche, Ausleihe oder Bewertung ohne zusätzliche Anleitung verständlich und ausführbar sind.}
	\nfr{Das System muss sicherstellen, dass nur autorisierte Clients auf geschützte Endpunkte wie \texttt{/api/books/secure/**} oder \texttt{/api/admin/secure/**} zugreifen können.}
	\nfr{Die Anwendung muss HTTP-Methoden wie POST, PUT, DELETE und PATCH für bestimmte Ressourcen (z.\,B.\ \texttt{Book}, \texttt{Review}, \texttt{Message}) deaktivieren, wenn sie nicht benötigt werden.}
	\nfr{Die Anwendung muss Sicherheitsstandards einhalten, indem sie CSRF-Schutz für REST-APIs deaktiviert, wo dieser nicht erforderlich ist, insbesondere bei stateless JWT-Authentifizierung.}
	
	
	\caption{Nicht-funktionale Anforderungen von LibraNova}
	\label{tab:non-functional-requirements}
\end{longtable}



\subsection{Technische  Anforderungen}

Technische Anforderungen beschreiben, welche Technologien, Werkzeuge und Methoden für die Entwicklung und den Betrieb von LibraNova verwendet werden. Dazu gehören zum Beispiel Programmiersprachen, Frameworks, Schnittstellen und Protokolle.\\ \\
Die folgende Tabelle \ref{tab:technical-requirements} zeigt die wichtigsten technischen Anforderungen von LibraNova.
\begin{longtable}{|c|p{13cm}|}
	\hline
	\textbf{ID} & \textbf{Anforderung (Beschreibung)} \\
	\hline
	\endfirsthead
	
	\hline
	\textbf{ID} & \textbf{Anforderung (Beschreibung)} \\
	\hline
	\endhead
	
	\tr{Die Anwendung muss Okta als Identity Provider integrieren und dabei OAuth2, OpenID Connect und JWT verwenden.}
	\tr{Das Backend muss HTTPS-Verbindungen über Port 8443 ermöglichen, mithilfe eines selbstsignierten SSL-Zertifikats.}
	\tr{Der Zugriff auf geschützte API-Endpunkte muss über JWT-Token abgesichert werden (z.,B.\ /api/books/secure/**}
	\tr{CSRF-Schutz muss für REST-APIs deaktiviert werden, da die Authentifizierung stateless über JWT erfolgt.}
	\tr{Die Anwendung muss rollenbasierte Zugriffskontrolle implementieren (Nutzer vs. Administrator).}
	\tr{Das Backend muss eine RESTful API zur Verfügung stellen, um Daten zwischen Frontend und Backend zu übertragen.}
	\tr{Die API-Endpunkte müssen CORS-konform konfiguriert sein und ausschließlich Anfragen von https://localhost:3000 akzeptieren.}
	\tr{Die REST-API muss mit Swagger dokumentiert sein und über Swagger UI erreichbar sein.}
	\tr{Die Anwendung muss HTTP-Methoden wie POST, PUT, DELETE und PATCH für bestimmte Ressourcen (z.\,B.\ \texttt{Book}, \texttt{Review}, \texttt{Message}) deaktivieren, wenn sie nicht benötigt werden.}
	\tr{Die Anwendung muss eine relationale MySQL-Datenbank verwenden (libranova\_db), angebunden über JDBC.}
	\tr{Die Anwendung muss Spring Data JPA zur Datenpersistierung verwenden.}
	\tr{Die Anwendung muss Entities wie Book, Review und Message als JPA-Entities modellieren und mittels Repository-Schnittstellen verfügbar machen.}
	\tr{Die Anwendung muss Spring Data REST verwenden, um Repository-Schnittstellen automatisch als RESTful API verfügbar zu machen.}
	\tr{Lombok wird zur Reduzierung von Boilerplate-Code in Entities und anderen Klassen verwendet.}
	\tr{Die Anwendung muss Spring Boot Web Starter verwenden, um Web-Server-Funktionalitäten und REST-Controller bereitzustellen.}
	\tr{Die Anwendung muss Unit-Tests mit JUnit und Mocking mit Mockito implementieren, um die Funktionalität des Backends abzusichern.}
	\tr{Das Frontend muss in React umgesetzt sein und via HTTPS über Port 3000 laufen.}
	\tr{Das Frontend muss mit dem Backend über REST-APIs kommunizieren.}
	\tr{Die Benutzeranmeldung muss über das Okta Sign-In Widget erfolgen.}
	\tr{Das Frontend muss die vom Backend bereitgestellten Endpunkte nutzen und Authentifizierung über Tokens abwickeln.}
	
	
	
	\caption{Technische Anforderungen von LibraNova}
	\label{tab:technical-requirements}
\end{longtable}

\chapter{Methodologie und Systementwurf }



\section{Front-End Technologien}\index{Front-End Technologien}


\section{Back-End Technologien}\index{Back-End Technologien}


\subsubsection{Absicherung der REST-Endpunkte mit Spring Security}

Zur Absicherung der REST-Endpunkte in „LibraNova“ wurde das Framework \textbf{Spring Security} in Kombination mit OAuth2 und JSON Web Tokens (JWT) eingesetzt. Dabei schützt eine zentrale Sicherheitskonfiguration gezielt sensible Pfade wie \texttt{/api/books/secure/\***}, Alle übrigen Endpunkte bleiben öffentlich zugänglich. Diese Trennung zwischen geschützten und öffentlichen Routen erlaubt eine kontrollierte Zugriffskontrolle und gewährleistet gleichzeitig Offenheit für nicht sensible Daten. \\ 
Die Konfiguration erfolgt über eine \texttt{SecurityFilterChain}-Bean. Hierbei wurde die CSRF-Absicherung deaktiviert, da sie bei stateless JWT-Authentifizierung überflüssig ist. Gleichzeitig wird eine OAuth2-Resource-Server-Konfiguration mit JWT-Validierung verwendet. Die Einrichtung einer \texttt{ContentNegotiationStrategy} unterstützt eine saubere Inhaltsaushandlung zwischen Client und Server. Die Integration der \texttt{Okta.configureResourceServer401ResponseBody(http)}-Methode verbessert zudem die Fehlerbehandlung bei unautorisierten Zugriffen.

\begin{lstlisting}[language=Java, caption={Spring Security-Konfiguration}]
	http
	.csrf(csrf -> csrf.disable())
	.authorizeHttpRequests(configurer -> configurer
	.requestMatchers("/api/books/secure/**", 
	"/api/reviews/secure/**", 
	"/api/messages/secure/**", 
	"/api/admin/secure/**").authenticated()
	.anyRequest().permitAll())
	.oauth2ResourceServer(oauth2 -> oauth2.jwt())
	.cors(cors -> cors.configurationSource(corsConfigurationSource()));
\end{lstlisting}
Durch diese Konfiguration wird sichergestellt, dass nur authentifizierte Nutzer mit gültigem JWT-Token auf geschützte Ressourcen zugreifen können, während gleichzeitig der Zugriff auf öffentliche Inhalte uneingeschränkt möglich bleibt.

\subsubsection{CORS-Konfiguration und Einschränkung von HTTP-Methoden}

Zur zusätzlichen Absicherung des Backends wurden zwei Maßnahmen umgesetzt: Erstens wurde eine \textbf{CORS-Konfiguration} implementiert, die ausschließlich Anfragen vom React-Frontend (\texttt{https://localhost:3000}) erlaubt. Zweitens wurden mithilfe von \texttt{RepositoryRestConfigurer} bestimmte HTTP-Methoden wie \texttt{POST}, \texttt{PUT}, \texttt{PATCH} und \texttt{DELETE} auf ausgewählten Entitäten deaktiviert, um ungewollte Änderungen über Spring Data REST zu verhindern.

\begin{lstlisting}[language=Java, caption={CORS und HTTP-Methodenbeschränkung}]
	config.exposeIdsFor(Book.class, Review.class, Message.class);
	
	HttpMethod[] unsupported = {POST, PUT, PATCH, DELETE};
	restrictHttpMethods(Book.class, config, unsupported);
	
	corsRegistry.addMapping(config.getBasePath() + "/**")
	.allowedOrigins("https://localhost:3000");
\end{lstlisting}
Diese Konfiguration trägt maßgeblich zur Sicherheit und Stabilität der Anwendung bei, indem sie sowohl die erlaubten Ursprünge als auch die zulässigen Zugriffsarten explizit definiert.



\subsubsection{SSL-Konfiguration im Backend}

Zur Absicherung des Datenverkehrs wurde in der Anwendung eine HTTPS-Konfiguration implementiert, bei der der Server auf Port \texttt{8443} HTTPS-Anfragen entgegennimmt (siehe \ref{HTTPS-Config}). Dafür wurde SSL/TLS aktiviert, um eine verschlüsselte Kommunikation zwischen Client und Server zu gewährleisten. \\
Die Verschlüsselung basiert auf einem selbstsignierten SSL-Zertifikat, das mit dem folgenden Befehl generiert wurde:
\begin{lstlisting}[language=bash, caption={Generierung eines SSL-Zertifikats}, breaklines=true]
	keytool -genkeypair -alias libranova \
	-keystore src/main/resources/libranova-keystore.p12 \
	-keypass secret -storepass secret -storeType PKCS12 \
	-keyalg RSA -keysize 2048 -validity 365 \
	-dname "C=DE, ST=Rhineland-Palatinate, L=Trier, O=libranova, OU=Studies Backend, CN=localhost" \
	-ext "SAN=dns:localhost"
\end{lstlisting}
Dabei wurde eine Keystore-Datei im \texttt{PKCS12}-Format (\texttt{libranova-keystore.p12}) erstellt, in der das Zertifikat unter dem Alias \texttt{libranova} gespeichert ist. Diese Datei wird in der \texttt{application.properties} wie folgt eingebunden:

\begin{lstlisting}[language=, label=HTTPS-Config, caption={HTTPS- und SSL-Konfiguration}]
	# HTTPS settings
	server.port=8443
	server.ssl.enabled=true
	server.ssl.key-alias=libranova
	server.ssl.key-store=classpath:libranova-keystore.p12
	server.ssl.key-store-password=secret
	server.ssl.key-store-type=PKCS12
\end{lstlisting}
Durch diese Konfiguration wird sichergestellt, dass alle übermittelten Daten verschlüsselt und vor unbefugtem Zugriff geschützt sind. Die Anwendung erfüllt damit grundlegende Sicherheitsanforderungen moderner Webanwendungen.


\subsubsection{SSL-Konfiguration im Frontend}

Auch im Frontend wurde eine lokale HTTPS-Verbindung eingerichtet, um eine verschlüsselte Kommunikation mit dem Backend zu ermöglichen. Dazu wurde mithilfe von OpenSSL ein selbstsigniertes Zertifikat erstellt:

\begin{lstlisting}[language=bash, caption={Generierung eines Frontend-Zertifikats}]
	openssl req -x509 \
	-out ssl-localhost-libranova/localhost.crt \
	-keyout ssl-localhost-libranova/localhost.key \
	-newkey rsa:2048 -nodes -sha256 -days 365 \
	-config localhost.conf
\end{lstlisting}
Dieses Kommando erzeugte zwei Dateien:
\begin{itemize}
	\item \texttt{localhost.crt} – das Zertifikat
	\item \texttt{localhost.key} – der zugehörige private Schlüssel
\end{itemize}
In der \texttt{.env}-Datei wurden diese Dateien anschließend referenziert, um die React-Entwicklungsumgebung über HTTPS zu betreiben:

\begin{lstlisting}[language=bash, caption={.env Konfiguration für HTTPS und API-Zugriff}]
	SSL_CRT_FILE=ssl-localhost-libranova/localhost.crt
	SSL_KEY_FILE=ssl-localhost-libranova/localhost.key
	REACT_APP_API_URL='https://localhost:8443/api'
\end{lstlisting}
Die Konfigurationsdatei \texttt{localhost.conf} enthielt die benötigten Informationen für das Zertifikat (wie Standort und Common Name), um den Generierungsprozess zu automatisieren:

\begin{lstlisting}[caption={localhost.conf}]
	[req]
	prompt = no
	distinguished_name = dn
	
	[dn]
	C = DE
	ST = Rhineland-Palatinate
	L = Trier
	O = Libranova
	OU = Studies
	CN = localhost
\end{lstlisting}
Diese Konfiguration ermöglicht eine sichere Verbindung zwischen Frontend und Backend während der lokalen Entwicklung. \\ \\
Die HTTPS-Implementierung mit SSL/TLS gewährleistet die Einhaltung von Sicherheitsstandards und schützt sensible Nutzerdaten zuverlässig.

















\chapter{Implementierung}\label{Bausteine}

In diesem Kapitel wird die Implementierung und die prinzipielle Funktionsweise der Anwendung beschrieben.

\section{Architektur und Projektstruktur}\index{Abschnitt}\label{Architektur und Projektstruktur}

\subsection{Backend-Struktur}

\subsection{Frontend-Struktur}

\section{Backend-Implementierung}\index{Abschnitt}\label{Backend-Implementierung}

\subsection{Spring Boot Hauptklasse und Konfiguration}

\subsection{REST API Endpunkte}

\subsection{Datenzugriff}

\section{Frontend-Implementierung}\index{Abschnitt}\label{Frontend-Implementierung}

\subsection{Hauptkomponenten und Routing}

\subsection{Authentifizierung und Autorisierung im Frontend}

\subsection{State Management}

\section{Teststrategien und Qualitätssicherung}\index{Abschnitt}\label{Teststrategien und Qualitätssicherung}

\subsection{Backend-Tests}

\subsection{Frontend-Tests}

\chapter{Ergebnisse und Analyse}

\section{Allgemeine Benutzeroberfläche}

\subsection{Header und Navigation}

\subsection{Footer}

\subsection{Startseite (Main Page)}

\subsubsection{Karussell (Carousel)}

\subsubsection{Links und Informationen}

\section{Seitenübersicht}

\subsection{Buchsuche (Search Page)}

\subsection{Buchdetails (Book Page)}

\subsection{Rezensionsseite (Reviews Page)}

\section{Benutzerbezogene Funktionen}

\subsection{Bibliotheksaktivität (Library Activity)}

\subsubsection{Ausleihen (Loans)}

\subsubsection{Ausleihhistorie (History)}

\subsection{Bibliotheksservice (Library Service)}

\section{Admin-Bereich}

\subsection{Buchverwaltung}

\subsection{Benutzeranfragenverwaltung}


\chapter{Resümee und Ausblick}

In dieser Arbeit wurde die Konzeption und Realisierung einer modernen, benutzerfreundlichen und sicheren Bibliotheksmanagement-Plattform namens LibraNova vorgestellt. Das Ziel war es, eine skalierbare und wartbare Lösung zu entwickeln, die sowohl den Anforderungen der Nutzerinnen und Nutzer als auch der Bibliotheksadministration gerecht wird. Dabei stand die Integration aktueller Webtechnologien wie Spring Boot im Backend, React mit TypeScript im Frontend sowie eine umfassende Sicherheitsarchitektur mit Okta, JWT, OAuth2 und OpenID Connect im Fokus. Zudem wurde die Plattform um eine Zahlungsfunktion erweitert, die die Stripe API nutzt, um beispielsweise etwaige Gebühren reibungslos und sicher abzuwickeln.

\noindent Die Anwendung ist mehrsprachig ausgelegt und unterstützt sowohl Deutsch als auch Englisch, um eine breitere Nutzerbasis anzusprechen. Darüber hinaus legt das System großen Wert auf responsives Design, sodass die Plattform auf verschiedenen Endgeräten — sei es Desktop, Tablet oder Smartphone — optimal genutzt werden kann. Die Umsetzung erfolgte durch modulare Architekturen, die REST-APIs nutzen, um eine flexible Anpassung an verschiedene Nutzungsszenarien zu ermöglichen. Wesentliche Funktionen wie die Buchsuche, das Ausleihmanagement, Nutzerbewertungen und die administrative Verwaltung wurden effizient implementiert und durch systematische Tests abgesichert.

\noindent Die Plattform unterstützt zudem zukünftige Erweiterungen, etwa Cloud-Bereitstellung, erweiterte Testabdeckung sowie personalisierte Empfehlungen und Benachrichtigungen. Mit LibraNova wurde eine innovative Lösung geschaffen, die den digitalen Wandel im Bibliothekswesen aktiv mitgestaltet und die Nutzererfahrung durch intuitive Bedienung, hohe Sicherheitsstandards sowie eine responsive Gestaltung deutlich verbessert. Das Projekt bildet eine solide Grundlage für zukünftige Entwicklungen und bietet zahlreiche Potenziale zur weiteren Optimierung und Erweiterung.


%------------------ Literaturverzeichnis & Index -------------------------------
\backmatter
\bibliography{literatur}								% Literaturverzeichnis (literatur.bib)
\printindex												% Index (optional)


%------------------ Anhänge ----------------------------------------------------
\begin{appendix}
	\include{chapters/Abkürzungsverzeichnis}							% Glossar (optional)
	\include{chapters/Selbststaendigkeitserklaerung}	% Selbstständigkeitserklärung
\end{appendix}


\end{document}
