\chapter{Eigenständigkeitserklärung}

% Falls die vorliegende Arbeit als Gruppenarbeit angefertigt wurde, muss diese Eigenständigkeitserklärung dupliziert werden und von jedem auf dem Deckblatt angegebenen Bearbeiter separat ausgefüllt werden.

\begin{small}

\begin{description}
\item[$\Box$] Die vorliegende Arbeit wurde als Einzelarbeit angefertigt.\\ %check

\item[$\Box$] Die vorliegende Arbeit wurde als Gruppenarbeit angefertigt. Mein Anteil an der Gruppenarbeit ist im untenstehenden Abschnitt \emph{Verantwortliche} dokumentiert:\\

\vspace{1cm}

\item[$\Box$] Hiermit erkläre ich, dass ich die vorliegende Arbeit selbstständig und ohne unzulässige Hilfe Dritter angefertigt habe. Ich habe keine anderen als die angegebenen Quellen und Hilfsmittel benutzt sowie wörtliche und sinngemäße Zitate als solche kenntlich gemacht. Darüber hinaus erkläre ich, dass ich die vorliegende Arbeit in dieser oder ähnlicher Form noch nicht als Prüfungsleistung eingereicht habe.\\ %check

\vspace{1cm}

\item[$\Box$] Es ist keine Nutzung von KI-basierten text- oder inhaltgenerierenden Hilfsmitteln erfolgt.\\

\item[$\Box$] Die Nutzung von KI-basierten text- oder inhaltgenerierenden Hilfsmitteln wurde von der/dem Prüfenden ausdrücklich gestattet. Die von der/dem Prüfenden mit Ausgabe der Arbeit vorgegebenen Anforderungen zur Dokumentation und Kennzeichnung habe ich erhalten und eingehalten. Sofern gefordert, habe ich in der untenstehenden Tabelle \emph{Nutzung von KI-Tools} die verwendeten KI-basierten text- oder inhaltgenerierenden Hilfsmittel aufgeführt und die Stellen in der Arbeit genannt. Die Richtigkeit übernommener KI-Aussagen und Inhalte habe ich nach bestem Wissen und Gewissen überprüft.\\ %check
\end{description}

\vspace{4cm}
\begin{minipage}[t]{3cm}
	\rule{3cm}{0.5pt}
	Datum
\end{minipage}
\hfill
\begin{minipage}[t]{9cm}
	\rule{9cm}{0.5pt}
	Unterschrift der Kandidatin/des Kandidaten
\end{minipage}

\end{small}

\newpage

\section*{Verantwortliche}

Der alleinige Autor und Verantwortliche für sämtliche Kapitel der vorliegenden Dokumentation sowie den Quellcode ist:

\noindent \textbf{Mohammed Salih Mezraoui}


\newpage

\subsection*{Nutzung von KI-Tools}

\begin{table}
	\begin{small}	
		\begin{tabularx}{\textwidth}{|p{2.5cm}|p{3cm}|X|X|X|X|}
			\hline		
			\textbf{KI-Tool} & \textbf{Genutzt für} & \textbf{Warum?} & \textbf{Wann?} & \textbf{Mit welcher Eingabefrage bzw. -aufforderung?} & \textbf{An welcher Stelle der Arbeit übernommen?}\\
			\hline
			ChatGPT & Zusammenfassungen und einleitende Absätze & Unterstützung beim Formulieren von verständlichen Zusammenfassungen und Einleitungen & Während der Erstellung der Kapitel & „Ist es eine bessere Formulierung für diesen Abschnitt?“ / „Wie kann ich diese beiden Ideen zu einem Abschnitt kombinieren?“ / „Welche Reihenfolge der Abschnitte ist sinnvoll?“ / „Auf welchen Punkt sollte ich mich hier konzentrieren?“& Kapitel \emph{Einleitung}, Kapitel \emph{Verwandte Arbeiten}, Kapitel \emph{Anwendungsszenarien}\\
			\hline
			ChatGPT  & Fehlerbehebung und Installationshinweise & Hilfe beim Lösen von Laufzeit- und Exceptions-Fehlern sowie Installationshinweisen  & Während der Implementierung der Anwendung & „Wie behebe ich diesen Fehler?“ / „Wie installiere ich Bibliothek X?“  & Bei jedem Auftreten eines Fehlers während der Implementierung\\
			\hline
			DeepL Write & Korrektur und Übersetzung von Texten & Verbesserung der Lesbarkeit und Übersetzung von Quellen ins Deutsche & Während der Erstellung der Kapitel & „Formuliere diesen Text in korrektes Deutsch“ / „Übersetze den englischen Text sinngemäß ins Deutsche“ & Zusammenfass-ungen, Abstract, Kapitel \emph{Grundlagen}, Kapitel \emph{Resümee und Ausblick}\\
			\hline
		\end{tabularx}
	\end{small}
\end{table}


