\chapter{Einleitung}

In einer zunehmend digitalen Welt wird die effiziente Verwaltung von Informationen und Ressourcen immer wichtiger. Bibliotheken – als zentrale Einrichtungen der Wissensvermittlung – stehen vor der Herausforderung, ihre Prozesse zu modernisieren und den veränderten Anforderungen der Nutzer:innen gerecht zu werden. Webbasierte Anwendungen bieten großes Potenzial, um den Zugang zu Medien, die Organisation von Ausleihen und die Kommunikation zwischen Nutzer:innen und Verwaltung zeitgemäß und benutzerfreundlich zu gestalten. 

\noindent Vor diesem Hintergrund wurde die Anwendung \textit{LibraNova} entwickelt. Ziel ist es, eine moderne Full-Stack-Webanwendung bereitzustellen, die Bibliotheksprozesse effizient, sicher und barrierefrei digitalisiert – sowohl für Nutzer:innen als auch Administrator:innen. 

\noindent Im Folgenden werden die Motivation und die Ziele dieser Arbeit vorgestellt.


\section{Motivation}\index{Motivation}

\textit{LibraNova} adressiert genau diese Schwachstellen: Die Anwendung soll zeigen, wie durch moderne Webtechnologien – wie \textbf{Spring Boot}, \textbf{React} und \textbf{TypeScript} – eine skalierbare, intuitive und sichere Lösung für das Bibliotheksmanagement geschaffen werden kann. Besondere Schwerpunkte liegen auf der Mehrsprachigkeit (Deutsch und Englisch), Barrierefreiheit und responsiven Benutzerführung, um eine möglichst breite Zielgruppe anzusprechen und digitale Teilhabe aktiv zu fördern.


\section{Ziele der Arbeit}\index{Ziele der Arbeit}

Das Ziel dieser Arbeit besteht in der Konzeption, Umsetzung und Dokumentation einer webbasierten Bibliotheksanwendung, die folgende Kernanforderungen erfüllt:

\begin{itemize}
	\item Bereitstellung einer intuitiven Benutzeroberfläche für die Recherche, Ausleihe und Bewertung von Büchern.
	\item Ermöglichung einer effizienten Verwaltung des Buchbestands durch Administrator:innen.
	\item Integration moderner Authentifizierungs- und Autorisierungsmechanismen (\textbf{JWT}, \textbf{OAuth2} via Okta).
	\item Einbindung eines sicheren Zahlungssystems über die \textbf{Stripe API} für kostenpflichtige Dienste.
	\item Mehrsprachige Bereitstellung der Anwendung (Deutsch und Englisch).
	\item Berücksichtigung von Aspekten der Barrierefreiheit zur Förderung inklusiver Nutzung.
	\item Umsetzung eines \textbf{responsiven Designs} zur optimalen Darstellung auf verschiedenen Endgeräten (Desktop, Tablet, Smartphone).
	\item Sicherstellung von Softwarequalität durch automatisierte Tests mit \textbf{JUnit} und \textbf{Mockito}.  
\end{itemize}  

\noindent Darüber hinaus soll die Anwendung modular und erweiterbar gestaltet sein, um zukünftige Funktionalitäten problemlos integrieren zu können. Durch diese Arbeit soll ein Beitrag zur praxisnahen Entwicklung moderner Webanwendungen im Bildungs- und Informationsbereich geleistet werden.