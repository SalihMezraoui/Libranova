\chapter{Einleitung und Anforderungen}
In diesem Kapitel werden zentrale Aspekte der Anwendung „LibraNova“ vorgestellt. Der Fokus liegt dabei auf:

\begin{itemize}
	\item Einem kompakten Überblick über die Ziele und Funktionen der entwickelten Bibliotheksanwendung.
	\item Sowie der Erhebung und Beschreibung der funktionalen, nicht-funktionalen und technischen Anforderungen, die im Rahmen der Entwicklung berücksichtigt wurden.
\end{itemize}

\section{Überblick}\index{Überblick}

„LibraNova“ ist ein modernes Full-Stack-Bibliotheksmanagementsystem, das eine einfache und effiziente Verwaltung von Buchausleihen ermöglicht. Nutzer können Bücher bequem suchen, ausleihen, bewerten und ihre Ausleihen verfolgen – alles über eine benutzerfreundliche und für mobile Geräte optimierte Webanwendung. Die Plattform funktioniert zuverlässig auf verschiedenen Geräten und Bildschirmgrößen.\\
Neben der übersichtlichen Oberfläche bietet „LibraNova“ wichtige Funktionen wie eine Echtzeit-Anzeige der Buchverfügbarkeit, ein Bewertungssystem sowie eine sichere Nutzeranmeldung mit JWT und OAuth2. Administratoren können Bücher verwalten und auf Nutzeranfragen reagieren. Die Integration externer Dienste wie Stripe ermöglicht sichere Zahlungen für kostenpflichtige Funktionen. Technologisch basiert die Anwendung auf Spring Boot, React und MySQL und ist somit gut skalierbar, sicher und leistungsfähig.

\section{Anforderungen}\index{Anforderungen}

Die folgenden Unterabschnitte beschreiben die zentralen Anforderungen an das System. Dazu zählen funktionale, nicht-funktionale sowie technische Anforderungen, die für eine strukturierte Umsetzung der Anwendung erforderlich sind.

\subsection{Funktionale Anforderungen}

Funktionale Anforderungen definieren, welche Dienste das System leisten soll und wie es sich bei bestimmten Eingaben oder in bestimmten Situationen verhalten soll.\\ \\
Die Tabelle \ref{tab:functional-requirements} zeigt alle funktionalen Anforderungen der Anwendung.

\begin{longtable}{|c|p{13cm}|}
	\hline
	\textbf{ID} & \textbf{Anforderung (Beschreibung)} \\
	\hline
	\endfirsthead
	
	% No header repetition on next pages
	\endhead
	
	% Step 2: Use \fr{...} to add entries
	\fr{Nutzer können Bücher nach Titel oder Kategorie suchen.}
	\fr{Nutzer können die Verfügbarkeit eines Buches in Echtzeit einsehen.}
	\fr{Nutzer können bis zu fünf Bücher gleichzeitig ausleihen.}
	\fr{Nutzer können ihre ausgeliehenen Bücher einsehen.}
	\fr{Nutzer können ihre Ausleihhistorie einsehen.}
	\fr{Nutzer können ein Buch bewerten.}
	\fr{Nutzer können eine Rezension verfassen.}
	\fr{Nutzer können Rezensionen anderer Nutzer lesen.}
	\fr{Das System zeigt eine Übersicht der verfügbaren Bücher im Katalog an.}
	\fr{Administratoren können neue Bücher zum Katalog hinzufügen.}
	\fr{Administratoren können die Anzahl der Exemplare eines vorhandenen Buches erhöhen oder verringern.}
	\fr{Administratoren können Bücher aus dem Katalog löschen.}
	\fr{Nutzer können sich in ihr persönliches Konto einloggen.}
	\fr{Das System verwaltet Nutzerrollen (z.\,B. Nutzer, Administrator).}
	\fr{Nutzer können die Leihfrist eines Buches um bis zu 7 Tage verlängern.}
	\fr{Nutzer können ausgeliehene Bücher zurückgeben.}
	\fr{Das System zeigt an, wenn ein Buch nicht verfügbar ist.}
	\fr{Für kostenpflichtige Dienstleistungen ist eine Zahlungsabwicklung über Stripe integriert.}
	\fr{Nutzer müssen eingeloggt sein, um ein Buch auszuleihen.}
	\fr{Nutzer müssen eingeloggt sein, um eine Rezension verfassen zu können.}
	\fr{Nutzer müssen eingeloggt sein, um ein Buch bewerten zu können.}
	\fr{Das System zeigt an, wie viele Bücher von den maximal fünf gleichzeitig ausgeliehen sind.}
	\fr{Das System zeigt die Anzahl der Suchergebnisse basierend auf eingegebenen Suchbegriffen und gewählten Kategorien an.}
	\fr{Das System zeigt Bücherlisten (Suchergebnisse, ausgeliehene Bücher, Ausleihhistorie) paginiert an.}
	\fr{Das System muss sicherstellen, dass Nutzer sowohl einen Nachrichtentitel als auch den Nachrichtentext angeben, bevor sie eine Anfrage absenden können.}
	\fr{Das System muss sicherstellen, dass alle erforderlichen Felder beim Anlegen eines neuen Buches ausgefüllt sind.}
	\fr{Das System muss sicherstellen, dass Administratoren das Antwortfeld ausfüllen, bevor sie eine Antwort absenden können.}
	\fr{Das System muss im Ausleihverlauf eines Nutzers sowohl das Ausleihdatum als auch das Rückgabedatum jedes Buches anzeigen.}
	
	
	\caption{Funktionale Anforderungen von LibraNova}
	\label{tab:functional-requirements}
\end{longtable}



\subsection{Nich-funktionale Anforderungen}
Nicht-funktionale Anforderungen beschreiben die Qualitätsmerkmale und Randbedingungen des Systems, wie etwa Leistung, Sicherheit, Benutzbarkeit und Zuverlässigkeit\\ \\
Die folgende Tabelle \ref{tab:non-functional-requirements} fasst die wichtigsten nicht-funktionalen Anforderungen für LibraNova zusammen.

\begin{longtable}{|c|p{13cm}|}
	\hline
	\textbf{ID} & \textbf{Anforderung (Beschreibung)} \\
	\hline
	\endfirsthead
	
	\hline
	\textbf{ID} & \textbf{Anforderung (Beschreibung)} \\
	\hline
	\endhead
	
	\nfr{Das System muss eine responsive Benutzeroberfläche bieten, die auf verschiedenen Geräten und Bildschirmgrößen funktioniert.}
	\nfr{Die Anwendung muss eine sichere Kommunikation über HTTPS (TLS) gewährleisten, um die Datenübertragung zwischen Frontend und Backend zu schützen. (Ports 8443 und 3000).}
	\nfr{Die Authentifizierung und Autorisierung erfolgt über Okta als Identity Provider unter Verwendung von OAuth2, OpenID Connect und JWT, um sichere und standardisierte Benutzerzugriffe zu gewährleisten.}
	\nfr{Das System muss Cross-Origin Resource Sharing (CORS) korrekt konfigurieren, um Anfragen nur von der vertrauenswürdigen Frontend-Domain \texttt{https://localhost:3000} zuzulassen.}
	\nfr{Das System muss sicherstellen, dass Benutzer nur mit gültigen und nicht abgelaufenen Tokens Zugriff auf geschützte Ressourcen erhalten.}
	\nfr{Die Reaktionszeit der Such- und Filterfunktionen im Frontend darf 2 Sekunden nicht überschreiten, um ein flüssiges Nutzererlebnis zu gewährleisten.}
	\nfr{Die Anwendung muss grundlegende Barrierefreiheitsanforderungen gemäß WCAG 2.1 (Level AA) erfüllen, um auch Nutzern mit Einschränkungen einen uneingeschränkten Zugang und eine barrierefreie Nutzung zu ermöglichen.}
	\nfr{Die REST-APIs müssen robust gegen fehlerhafte Eingaben sein und validierte Anfragen verarbeiten, um die Stabilität des Systems zu gewährleisten.}
	\nfr{Die Backend-Services sollen modular aufgebaut sein, um einfache Wartbarkeit und Erweiterbarkeit zu ermöglichen.}
	\nfr{Die REST-API-Dokumentation muss aktuell und für Entwickler leicht zugänglich sein (über Swagger UI).}
	\nfr{Die Anwendung soll eine klare Trennung zwischen Benutzerrollen (Nutzer, Administrator) gewährleisten und Zugriffsrechte strikt durchsetzen.}
	\nfr{Die Anwendung muss auf allen gängigen Webbrowsern wie Firefox, Chrome und Safari ohne Einschränkungen lauffähig sein.}
	\nfr{NFR31: Der Anmeldevorgang eines Nutzers (ab dem Klick auf den Login-Button bis zur erfolgreichen Authentifizierung und Weiterleitung zur Startseite) darf nicht länger als 3 Sekunden dauern.}
	\nfr{Die Benutzeroberfläche muss auf allen Seiten konsistente Navigationselemente, Schaltflächen-Designs und Layouts verwenden, um eine einheitliche Nutzererfahrung sicherzustellen.}
	\nfr{Die Benutzerführung muss intuitiv gestaltet sein, sodass typische Nutzeraktionen wie Buchsuche, Ausleihe oder Bewertung ohne zusätzliche Anleitung verständlich und ausführbar sind.}
	\nfr{Das System muss sicherstellen, dass nur autorisierte Clients auf geschützte Endpunkte wie \texttt{/api/books/secure/**} oder \texttt{/api/admin/secure/**} zugreifen können.}
	\nfr{Die Anwendung muss HTTP-Methoden wie POST, PUT, DELETE und PATCH für bestimmte Ressourcen (z.\,B.\ \texttt{Book}, \texttt{Review}, \texttt{Message}) deaktivieren, wenn sie nicht benötigt werden.}
	\nfr{Die Anwendung muss Sicherheitsstandards einhalten, indem sie CSRF-Schutz für REST-APIs deaktiviert, wo dieser nicht erforderlich ist, insbesondere bei stateless JWT-Authentifizierung.}
	
	
	\caption{Nicht-funktionale Anforderungen von LibraNova}
	\label{tab:non-functional-requirements}
\end{longtable}



\subsection{Technische  Anforderungen}

Technische Anforderungen beschreiben, welche Technologien, Werkzeuge und Methoden für die Entwicklung und den Betrieb von LibraNova verwendet werden. Dazu gehören zum Beispiel Programmiersprachen, Frameworks, Schnittstellen und Protokolle.\\ \\
Die folgende Tabelle \ref{tab:technical-requirements} zeigt die wichtigsten technischen Anforderungen von LibraNova.
\begin{longtable}{|c|p{13cm}|}
	\hline
	\textbf{ID} & \textbf{Anforderung (Beschreibung)} \\
	\hline
	\endfirsthead
	
	\hline
	\textbf{ID} & \textbf{Anforderung (Beschreibung)} \\
	\hline
	\endhead
	
	\tr{Die Anwendung muss Okta als Identity Provider integrieren und dabei OAuth2, OpenID Connect und JWT verwenden.}
	\tr{Das Backend muss HTTPS-Verbindungen über Port 8443 ermöglichen, mithilfe eines selbstsignierten SSL-Zertifikats.}
	\tr{Der Zugriff auf geschützte API-Endpunkte muss über JWT-Token abgesichert werden (z.,B.\ /api/books/secure/**}
	\tr{CSRF-Schutz muss für REST-APIs deaktiviert werden, da die Authentifizierung stateless über JWT erfolgt.}
	\tr{Die Anwendung muss rollenbasierte Zugriffskontrolle implementieren (Nutzer vs. Administrator).}
	\tr{Das Backend muss eine RESTful API zur Verfügung stellen, um Daten zwischen Frontend und Backend zu übertragen.}
	\tr{Die API-Endpunkte müssen CORS-konform konfiguriert sein und ausschließlich Anfragen von https://localhost:3000 akzeptieren.}
	\tr{Die REST-API muss mit Swagger dokumentiert sein und über Swagger UI erreichbar sein.}
	\tr{Die Anwendung muss HTTP-Methoden wie POST, PUT, DELETE und PATCH für bestimmte Ressourcen (z.\,B.\ \texttt{Book}, \texttt{Review}, \texttt{Message}) deaktivieren, wenn sie nicht benötigt werden.}
	\tr{Die Anwendung muss eine relationale MySQL-Datenbank verwenden (libranova\_db), angebunden über JDBC.}
	\tr{Die Anwendung muss Spring Data JPA zur Datenpersistierung verwenden.}
	\tr{Die Anwendung muss Entities wie Book, Review und Message als JPA-Entities modellieren und mittels Repository-Schnittstellen verfügbar machen.}
	\tr{Die Anwendung muss Spring Data REST verwenden, um Repository-Schnittstellen automatisch als RESTful API verfügbar zu machen.}
	\tr{Lombok wird zur Reduzierung von Boilerplate-Code in Entities und anderen Klassen verwendet.}
	\tr{Die Anwendung muss Spring Boot Web Starter verwenden, um Web-Server-Funktionalitäten und REST-Controller bereitzustellen.}
	\tr{Die Anwendung muss Unit-Tests mit JUnit und Mocking mit Mockito implementieren, um die Funktionalität des Backends abzusichern.}
	\tr{Das Frontend muss in React umgesetzt sein und via HTTPS über Port 3000 laufen.}
	\tr{Das Frontend muss mit dem Backend über REST-APIs kommunizieren.}
	\tr{Die Benutzeranmeldung muss über das Okta Sign-In Widget erfolgen.}
	\tr{Das Frontend muss die vom Backend bereitgestellten Endpunkte nutzen und Authentifizierung über Tokens abwickeln.}
	
	
	
	\caption{Technische Anforderungen von LibraNova}
	\label{tab:technical-requirements}
\end{longtable}
