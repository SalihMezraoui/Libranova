\chapter{Grundlagen}

In diesem Kapitel werden die zentralen Technologien vorgestellt, die für die Konzeption und Entwicklung der Bibliotheksanwendung \textit{LibraNova} von Bedeutung sind.

\section{Back-End Technologien}\index{Back-End Technologien}

In diesem Abschnitt werden die eingesetzten Back-End-Technologien vorgestellt, wobei der Fokus auf Spring Boot liegt – dem zentralen Framework zur Implementierung einer robusten, skalierbaren und sicheren serverseitigen Logik sowie RESTful APIs für die Anwendung \textit{LibraNova}.

\subsection{Java}\index{Java}

Java ist eine weit verbreitete, objektorientierte Programmiersprache, die für ihre Plattformunabhängigkeit, Stabilität und umfangreichen Standardbibliotheken bekannt ist. Aufgrund ihrer Vielseitigkeit und Leistungsfähigkeit wird sie häufig für die Entwicklung von Back-End-Anwendungen eingesetzt. In der Bibliotheksanwendung \textit{LibraNova} wird Java in der Version 17 verwendet, um eine robuste, wartbare und skalierbare serverseitige Logik bereitzustellen \cite{ORACLE2025a}.

\subsection{Spring Boot}\index{Spring Boot}

Spring Boot ist ein Framework, das auf dem Spring Framework aufbaut und speziell entwickelt wurde, um schnelle, effiziente und skalierbare Anwendungen zu erstellen. Es bietet eine Vielzahl von Features, die den Entwicklungsprozess beschleunigen und vereinfachen, insbesondere für Java-basierte Webanwendungen \cite{SPRINGBOOT2025a}.

\subsection*{Gründe für die Wahl von Spring Boot}
Spring Boot wurde für \textit{LibraNova} gewählt, da es die Entwicklung serverseitiger Anwendungen deutlich vereinfacht und beschleunigt. Entscheidende Vorteile sind:
\begin{itemize}
	\item \textbf{Schneller Entwicklungsstart:} Vorkonfigurierte Abhängigkeiten und Best Practices ermöglichen sofortiges Beginnen der Entwicklung.
	\item \textbf{Einfache Bereitstellung:} Eigenständig ausführbare Anwendungen mit eingebautem Tomcat-Server benötigen keinen externen Applikationsserver.
	\item \textbf{Automatische Konfiguration:} Frameworks und Bibliotheken werden automatisch erkannt, was den Konfigurationsaufwand reduziert.
	\item \textbf{Produktionsreife Funktionen:} Integrierte Features wie Konfigurationsdateien, Monitoring und Health Checks vereinfachen den produktiven Betrieb \cite{SPRINGBOOT2025b}.
\end{itemize}

\subsection*{Verwendete Abhängigkeiten}\index{verwendete Abhängigkeiten}

\begin{itemize}
	\item \textbf{Spring Data REST:} Ermöglicht die automatische Bereitstellung von RESTful Webservices auf Basis von Spring-Data-Repositories. Es erstellt durchsuchbare Endpunkte, unterstützt HAL, Paginierung, Sortierung und Filterung, ganz ohne manuelle Controller \cite{SPRINGREST2025}.
	
	\item \textbf{Spring Data JPA:} Vereinfacht die Implementierung von Repository-Schichten auf Basis der Java Persistence API. Entwickler definieren Interfaces, während Spring die Implementierung liefert. Unterstützt abgeleitete Suchmethoden, Paginierung, dynamische Abfragen und Auditierung von Entitäten \cite{SPRINGJPA2025}.
\end{itemize}


\subsection{Hibernate}\index{Hibernate}

Hibernate ist ein typensicheres Java-ORM-Framework, das automatische Abbildung zwischen Objekten und relationalen Datenbanken ermöglicht, JPA unterstützt, komplexe Abfragen, Transaktionen, Caching und Performance-Optimierung bietet. Es fördert idiomatische Java-Persistenz, HQL-Abfragen, Datenbankkompatibilität, Skalierbarkeit und Entwicklerproduktivität, reduziert Boilerplate-Code, gewährleistet ACID-Eigenschaften und vereinfacht die Entwicklung von Enterprise-Anwendungen \cite{HIBERNATE2025}.

\subsection{REST-API}\index{REST-API}

\noindent REST (Representational State Transfer) ist ein Architekturstil für verteilte Hypermedia-Systeme, der von Roy Fielding im Jahr 2000 in seiner Dissertation vorgestellt wurde. Seitdem hat sich REST als eine der am weitesten verbreiteten Methoden zur Entwicklung von webbasierten APIs etabliert \cite{GUPTA2025}. Es definiert zentrale Prinzipien, die ein Webservice erfüllen muss, um als RESTful zu gelten:

\begin{itemize}
	\item \textbf{Uniform Interface:} Ressourcen werden über eine einheitliche Schnittstelle eindeutig identifiziert und mittels standardisierter HTTP-Methoden (GET, POST, PUT, DELETE) manipuliert. Nachrichten sind selbstbeschreibend und können Hypermedia-Links enthalten.
	\item \textbf{Stateless:} Jeder Client-Request enthält alle notwendigen Informationen; der Server speichert keinen Sitzungszustand.
	\item \textbf{Client-Server-Architektur:} Trennung von Benutzeroberfläche und Datenhaltung ermöglicht unabhängige Weiterentwicklung und bessere Skalierbarkeit \cite{GUPTA2025}
\end{itemize}

\noindent Im \textit{LibraNova}-Projekt wird das Backend mit Spring Boot umgesetzt, um RESTful-Endpunkte für die Verwaltung verschiedener Ressourcen bereitzustellen. Die API-Pfade, beispielsweise \texttt{/api/books}, sind klar strukturiert, um einen konsistenten Zugriff auf die Daten zu gewährleisten.

\subsection{Spring Security}\index{Spring Security}

Spring Security ist ein anpassbares Framework zur Verwaltung von Authentifizierung und Autorisierung in Java-Anwendungen. Es bietet wiederverwendbare Module, Schutz gegen Web-Schwachstellen wie CSRF und flexible Integrationsmöglichkeiten für unterschiedliche Anwendungsfälle. Unsachgemäße Konfiguration kann jedoch Sicherheitsrisiken bergen  \cite{SPRNGSECURITY2025}.

\noindent Im \textit{LibraNova}-Projekt wird Spring Security eingesetzt, um die REST-API-Endpunkte abzusichern und den Zugriff zu kontrollieren.

\subsection{HTTPS und SSL/TLS}

HTTPS sichert die Kommunikation zwischen Browser und Webserver durch TLS-Verschlüsselung (früher SSL). Öffentlicher Schlüssel verschlüsselt Daten, die nur der private Schlüssel auf dem Server entschlüsseln kann. So werden sensible Informationen geschützt, und die Verbindung gewährleistet Vertraulichkeit, Integrität und Authentizität. Standardmäßig erfolgt die Kommunikation über Port 443 \cite{HTTPS2025}. 

\noindent In \textit{LibraNova} laufen sowohl das Backend als auch das Frontend über HTTPS in ihren jeweiligen Ports unter Verwendung eines selbstsignierten Zertifikats.

\subsection{Stripe API}\index{Stripe API}

Die Stripe API ermöglicht die nahtlose Integration von Zahlungsfunktionen in Anwendungen. Sie orientiert sich an REST-Prinzipien, verwendet ressourcenorientierte URLs, überträgt Anfragen im Formularformat und liefert Antworten im JSON-Format. Die Authentifizierung erfolgt über API-Schlüssel, und die API unterstützt sowohl Test- als auch Echtzeitmodi. Sie bietet umfassende Funktionen für Zahlungsabwicklung, Kundenverwaltung und Abo-Management sowie Zugriff auf das gesamte Stripe-Portfolio, einschließlich \textit{Payments}, \textit{Billing}, \textit{Connect}, \textit{Terminal}, \textit{Issuing} und \textit{Treasury}, wodurch individuelle Zahlungsprozesse und Automatisierungen effizient umgesetzt werden können \cite{STRIPEAPI2025}.

\noindent In  \textit{LibraNova} wurde Stripe eingesetzt, um eine moderne und sichere Zahlungsabwicklung zu ermöglichen.

\section{Front-End Technologien}\index{Front-End Technologien}
Dieser Abschnitt befasst sich mit den verwendeten Frontend-Technologien, darunter HTML, CSS, Bootstrap, TypeScript, React und i18next. Sie ermöglichen gemeinsam eine moderne, responsive und benutzerfreundliche Oberfläche für die Anwendung sowie eine effiziente Umsetzung der Internationalisierung.


\subsection{HTML/CSS}\index{HTML/CSS}
HTML und CSS bilden die Grundlage für die Entwicklung und Gestaltung von Webseiten. HTML definiert die Struktur und den Inhalt der Seite, während CSS für das visuelle Erscheinungsbild zuständig ist – einschließlich Layout, Farben und Schriftarten. Gemeinsam ermöglichen sie die Erstellung ansprechender und übersichtlich strukturierter Benutzeroberflächen \cite{HTMLCSS2025a, HTMLCSS2025b}.

\subsection{Bootstrap}\index{Bootstrap}
Bootstrap ist ein weit verbreitetes, quelloffenes Framework zur Entwicklung responsiver und mobiler Webanwendungen. Es stellt eine Vielzahl vordefinierter CSS-Klassen sowie JavaScript-Komponenten bereit, die eine schnelle und konsistente Gestaltung von Benutzeroberflächen ermöglichen \cite{BOOTSTRAP2025}. 

\noindent In \textit{LibraNova} wurde Bootstrap eingesetzt, um das Layout flexibel zu gestalten und sicherzustellen, dass sich die Benutzeroberfläche auf verschiedenen Bildschirmgrößen (z.\,B. Desktop, Tablet, Smartphone) dynamisch anpasst. 

\subsection{TypeScript}\index{TypeScript}
TypeScript wurde von Microsoft entwickelt und ist eine Programmiersprache, die JavaScript erweitert und optionale statische Typisierung sowie moderne Sprachfunktionen bietet. Sie ermöglicht es Entwickler:innen, viele häufige Fehler bereits während der Entwicklungsphase zu erkennen, wobei das ursprüngliche Laufzeitverhalten von JavaScript erhalten bleibt. Für die Entwicklung der React-basierten Benutzeroberfläche wurde bewusst TypeScript anstelle von reinem JavaScript gewählt. Die Gründe dafür liegen in der besseren Code-Wartbarkeit, der verbesserten Autovervollständigung in modernen IDEs und der höheren Typsicherheit, die gerade in größeren Anwendungen wie einem Bibliotheksverwaltungssystem eine zentrale Rolle spielt \cite{MICROSOFT2025, TYPESCRIPT2025}.



\subsection{React}\index{React}

React ist eine JavaScript-Bibliothek zum Erstellen von Benutzeroberflächen, indem kleine, wiederverwendbare Komponenten wie Buttons und Text kombiniert werden. Es verwendet einen deklarativen Ansatz, der nur die Teile der UI effizient aktualisiert, die sich ändern, wodurch der Code leichter verständlich und leichter zu debuggen ist. Komponenten kapseln ihre eigene Logik und ihren Zustand und können zu komplexen Oberflächen zusammengesetzt werden. Da die Komponenten in JavaScript geschrieben sind, ermöglichen sie einen reibungslosen Datenfluss ohne direkte Abhängigkeit vom DOM. Zur besseren Strukturierung können Komponenten in separate Dateien ausgelagert und bei Bedarf importiert werden. React ist besonders beliebt für die Entwicklung von Single-Page-Anwendungen (SPAs), die schnelle und nahtlose Nutzererlebnisse ohne vollständiges Neuladen der Seite bieten \cite{REACT2025a, REACT2025b, REACT2025c}. 

\noindent Die Kombination von React im Frontend mit Spring Boot im Backend ist weit verbreitet, da Spring Boot RESTful APIs bereitstellt, die React über HTTP-Anfragen konsumieren kann. Diese Trennung von Frontend und Backend unterstützt eine klare Architektur, fördert Skalierbarkeit und erleichtert die Wartung. Über REST APIs können dynamisch Daten wie Bücher, Benutzerinformationen oder Ausleihvorgänge in Echtzeit geladen und aktualisiert werden, was für eine Bibliotheksanwendung essenziell ist \cite{SPRINGBOOTREACT2025}. 

\subsection{i18next}\index{i18next}
i18next ist ein umfassendes Internationalisierungs-Framework für JavaScript, das Web-, Mobile- und Desktop-Plattformen unterstützt. Es bietet Funktionen wie automatische Spracherkennung, flexibles Laden von Übersetzungen und Erweiterbarkeit durch Plugins. Kompatibel mit wichtigen Frontend-Frameworks, ist es auf Skalierbarkeit und Benutzerfreundlichkeit ausgelegt. Diese Bibliothek wurde für \textit{LibraNova} aufgrund ihrer Popularität und ihres robusten Ökosystems ausgewählt \cite{i18NEXT2025}. 

\section{Datenbanktechnologien}\index{Datenbanktechnologien}
In diesem Abschnitt werden die wichtigsten Datenbanktechnologien erläutert, die für die Implementierung der \textit{LibraNova}-Anwendung verwendet wurden. Im Fokus stehen dabei MySQL als Datenbanksystem, MySQL Workbench als grafisches Verwaltungswerkzeug sowie SQL als zugrunde liegende Abfragesprache.

\subsection{MySQL}
MySQL ist ein weit verbreitetes Open-Source-Datenbankmanagementsystem, das Daten in relationalen Tabellen speichert und über SQL zugänglich macht. Es bietet hohe Geschwindigkeit, Zuverlässigkeit und Skalierbarkeit \cite{ORACLE2025b}.

\subsection{MySQL-Workbench und SQL}
MySQL Workbench ist ein grafisches Werkzeug zur Modellierung, Verwaltung und Migration von MySQL-Datenbanken. Es bietet Funktionen wie visuelles Datenbankdesign, Ausführung von SQL-Abfragen, Serveradministration sowie Datenmigration aus anderen Datenbanksystemen \cite{ORACLE2025d}. 

\noindent SQL entwickelt von Don Chamberlin und Ray Boyce bei IBM, ist eine standardisierte Sprache zur Verwaltung relationaler Datenbanken. Sie ermöglicht das Einfügen, Abfragen, Aktualisieren und Löschen von Daten mit wenigen Befehlen. Durch \texttt{JOIN}-Operationen können Daten aus mehreren Tabellen effizient verknüpft und Redundanzen vermieden werden. SQL ist ein Kernbestandteil moderner datengetriebener Anwendungen und kommt in nahezu allen relationalen Datenbanksystemen zum Einsatz \cite{IBM2025}. 

\section{Authentifizierungs- und Autorisierungsprotokolle}\index{Authentifizierungs- und Autorisierungsprotokolle}

Sichere Benutzerverwaltung und Zugriffskontrolle sind für Webanwendungen entscheidend. Technologien wie Okta, JWT, OAuth2 und OpenID Connect sorgen für standardisierte Authentifizierung und Autorisierung. Im Folgenden wird ihre Nutzung in der Anwendung beschrieben.

\subsection{Okta}
Okta ist ein cloudbasierter Identitätsdienst, der sicheren Zugriff auf Anwendungen und Geräte ermöglicht. Es bietet Single Sign-On (SSO), Multi-Faktor-Authentifizierung (MFA) und Integration mit lokalen Verzeichnissen wie Active Directory. Okta erleichtert das Identitäts- und Zugriffsmanagement über verschiedene Systeme hinweg \cite{OKTA2025}. Die Okta-Integration wird verwendet, um die sichere Benutzerverwaltung auszulagern, einschließlich Passwortverwaltung, Token-Ausgabe und rollenbasierter Zugriffskontrolle. Dadurch wird ein standardisierter Authentifizierungsprozess implementiert, der verschiedene Benutzerrollen unterstützt und Berechtigungen direkt in Okta verwaltet.

\subsection{JWT}

JWT ist ein offener Standard zur sicheren Übertragung von Informationen als signiertes JSON-Objekt. Es wird hauptsächlich für die Autorisierung genutzt, indem es nach der Anmeldung den Zugriff auf geschützte Ressourcen ermöglicht. Außerdem gewährleistet JWT die Integrität und Authentizität der übertragenen Daten \cite{JWT2025}. In der Anwendung sichern Okta-JWTs die API-Endpunkte. Spring Security ist als OAuth2-Resource-Server konfiguriert und validiert die Tokens, die Benutzer-Claims wie „userType“ enthalten, um rollenbasierte Zugriffssteuerung zu ermöglichen. So wird eine sichere, tokenbasierte Authentifizierung und feingranulare Autorisierung im Backend gewährleistet.


\subsection{OAuth2}
OAuth 2.0 ist ein Sicherheitsstandard, der Drittanbieteranwendungen ermöglicht, im Namen von Nutzern auf Ressourcen zuzugreifen, ohne deren Anmeldedaten preiszugeben. Es basiert auf dem Austausch von Zugriffstokens, was die Sicherheit erhöht \cite{OAUTH22025}. In der Anwendung wird OAuth 2.0 zusammen mit Okta genutzt, um JWTs auszustellen und zu validieren. Diese Tokens autorisieren den Zugriff auf geschützte Backend-Ressourcen, wodurch eine sichere und rollenbasierte Zugriffskontrolle gewährleistet wird.

\subsection{OpenID Connect}
OpenID Connect ist ein Authentifizierungsprotokoll, das auf OAuth 2.0 basiert und die sichere Überprüfung der Benutzeridentität ermöglicht. Es liefert standardisierte Nutzerinformationen und unterstützt eine einfache Integration in verschiedene Anwendungen \cite{OPENID2025}. Die Anwendung nutzt OpenID Connect mit Okta zur sicheren Anmeldung und zur Verwaltung von Benutzerrollen und Zugriffsrechten.

\section{Version Control mit Git und GitHub}\index{Version Control mit Git und GitHub}

Git ist ein kostenloses, verteiltes Versionskontrollsystem, das durch hohe Geschwindigkeit, effizientes Branch-Management und flexible Arbeitsabläufe besticht. Es ermöglicht die einfache Verwaltung von Projekten jeder Größe und unterstützt parallele Entwicklungsprozesse durch lokale Branches \cite{GIT2025}.

\noindent GitHub ergänzt Git als cloudbasierte Plattform zum Speichern, Teilen und gemeinsamen Entwickeln von Code. Es erleichtert das Nachverfolgen von Änderungen, die Code-Review durch andere Entwickler sowie die koordinierte Zusammenarbeit, ohne unbeabsichtigte Auswirkungen auf den Hauptzweig zu riskieren \cite{GITHUB2025b}. Beide Werkzeuge haben maßgeblich zur effizienten Verwaltung und Versionskontrolle des Codes im Verlauf dieses Projekts beigetragen.

\section{Testen und Qualitätssicherung}\index{Testen und Qualitätssicherung}

Zur Sicherstellung der Softwarequalität wurden im Projekt Unit-Tests, Mocking und API-Tests eingesetzt, um Funktionen, API-Endpunkte und Abhängigkeiten zuverlässig zu überprüfen.

\subsection{Unit-Tests mit JUnit 5}

JUnit 5 ist ein modernes Testframework für Java, das aus mehreren Modulen besteht: der JUnit Platform, JUnit Jupiter und JUnit Vintage. Die JUnit Platform bildet die Basis zur Ausführung von Tests auf der JVM und ermöglicht die Integration verschiedener Testengines. JUnit Jupiter bietet die Programmierschnittstelle und Erweiterungsmöglichkeiten für das Schreiben und Ausführen neuer Tests, während JUnit Vintage die Kompatibilität zu älteren JUnit-Versionen sicherstellt. JUnit 5 setzt mindestens Java 8 voraus und wird von gängigen Entwicklungsumgebungen und Build-Tools wie IntelliJ IDEA, Maven und Gradle unterstützt. Die modulare Architektur erleichtert sowohl das Schreiben als auch das Ausführen von Tests in modernen Java-Projekten \cite{JUNIT52025}.

\subsection{Mocking mit Mockito}

Mockito ist ein weit verbreitetes Mocking-Framework für Java, das speziell für das Schreiben von klaren und leicht lesbaren Unit-Tests entwickelt wurde. Es ermöglicht das einfache Erstellen von Scheinobjekten (Mocks), mit denen sich das Verhalten von Abhängigkeiten gezielt simulieren und testen lässt. Dank seiner übersichtlichen API trägt Mockito zu einer besseren Verständlichkeit der Tests und zu nachvollziehbaren Fehlermeldungen bei. Es wird von der Entwicklergemeinschaft intensiv genutzt und zählt zu den beliebtesten Bibliotheken im Java-Ökosystem \cite{MOCKITO2025}.

\subsection{API-Tests mit Postman}

Postman ist eine umfassende Plattform für die Arbeit mit APIs und unterstützt den gesamten Lebenszyklus – von der Planung über das Testen bis hin zur Bereitstellung und Überwachung. Mit einer intuitiven Oberfläche und zahlreichen Funktionen ermöglicht Postman das Speichern, Dokumentieren und Testen von API-Endpunkten in einem zentralen Repository. Es bietet Werkzeuge zur Spezifikation, Mock-Erstellung und Automatisierung von Tests, wodurch die Entwicklung beschleunigt und die Zusammenarbeit im Team vereinfacht wird \cite{POSTMAN2025}.

\section{Modellierung mit UML}\index{Modellierung mit UML}

UML ist ein wesentliches Werkzeug im Bereich Entwurf und ist eine standardisierte Modellierungssprache, die eine Vielzahl von Diagrammen bietet, um unterschiedliche Aspekte eines Systems darzustellen. Diese Diagramme helfen dabei, komplexe Systeme zu visualisieren, zu dokumentieren und zu kommunizieren \cite{UML2025}. Eine Auswahl relevanter Diagramme zur Konzeption der Webanwendung wird im folgenden Kapitel vorgestellt, um einen Überblick über den strukturellen Aufbau und das Zusammenspiel der Systemkomponenten zu geben.


                                                                                                                                                                                                                                                                                        