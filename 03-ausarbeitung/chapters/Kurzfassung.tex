\kurzfassung

Die Arbeit stellt „LibraNova“ vor – eine moderne und benutzerfreundliche Full-Stack-Webanwendung für das digitale Bibliotheksmanagement und die Ausleihe von Büchern. Entwickelt mit Spring Boot im Backend und React im Frontend, ermöglicht die Anwendung Nutzer:innen das Suchen von Büchern, das Prüfen ihrer Verfügbarkeit sowie das Lesen und Verfassen von Reviews. Administrator:innen können den Buchbestand verwalten und auf Anfragen der Nutzer:innen reagieren. Für Authentifizierung und Autorisierung kommen moderne Sicherheitsstandards wie Okta mit JWT, OAuth2 und OpenID Connect zum Einsatz. Die Zahlungsabwicklung kostenpflichtiger Dienste erfolgt über Stripe. Eine relationale MySQL-Datenbank gewährleistet eine strukturierte und effiziente Datenhaltung. Besonderer Fokus liegt auf intuitiver Benutzerführung, responsivem Design und automatisierten Tests (JUnit, Mockito) zur Qualitätssicherung.



