\chapter{Systemanforderungen}


\begin{longtable}{|c|p{13cm}|}
	\hline
	\textbf{ID} & \textbf{Anforderung (Beschreibung)} \\
	\hline
	\endfirsthead
	
	% No header repetition on next pages
	\endhead
	
	% Step 2: Use \fr{...} to add entries
	\fr{Nutzer können Bücher nach Titel oder Kategorie suchen.}
	\fr{Nutzer können die Verfügbarkeit eines Buches in Echtzeit einsehen.}
	\fr{Nutzer können bis zu fünf Bücher gleichzeitig ausleihen.}
	\fr{Nutzer können ihre ausgeliehenen Bücher einsehen.}
	\fr{Nutzer können ihre Ausleihhistorie einsehen.}
	\fr{Nutzer können ein Buch bewerten.}
	\fr{Nutzer können eine Rezension verfassen.}
	\fr{Nutzer können Rezensionen anderer Nutzer lesen.}
	\fr{Das System zeigt eine Übersicht der verfügbaren Bücher im Katalog an.}
	\fr{Administratoren können neue Bücher zum Katalog hinzufügen.}
	\fr{Administratoren können die Anzahl der Exemplare eines vorhandenen Buches erhöhen oder verringern.}
	\fr{Administratoren können Bücher aus dem Katalog löschen.}
	\fr{Nutzer können sich in ihr persönliches Konto einloggen.}
	\fr{Nutzer können die Leihfrist eines Buches um bis zu 7 Tage verlängern.}
	\fr{Nutzer können ausgeliehene Bücher zurückgeben.}
	\fr{Das System zeigt an, wenn ein Buch nicht verfügbar ist.}
	\fr{Für kostenpflichtige Dienstleistungen ist eine Zahlungsabwicklung über Stripe integriert.}
	\fr{Nutzer müssen eingeloggt sein, um ein Buch auszuleihen.}
	\fr{Nutzer müssen eingeloggt sein, um eine Rezension verfassen zu können.}
	\fr{Nutzer müssen eingeloggt sein, um ein Buch bewerten zu können.}
	\fr{Das System zeigt an, wie viele Bücher von den maximal fünf gleichzeitig ausgeliehen sind.}
	\fr{Das System zeigt die Anzahl der Suchergebnisse basierend auf eingegebenen Suchbegriffen und gewählten Kategorien an.}
	\fr{Das System zeigt Bücherlisten (Suchergebnisse, ausgeliehene Bücher, Ausleihhistorie) paginiert an.}
	\fr{Das System muss sicherstellen, dass Nutzer sowohl einen Nachrichtentitel als auch den Nachrichtentext angeben, bevor sie eine Anfrage absenden können.}
	\fr{Das System muss sicherstellen, dass alle erforderlichen Felder beim Anlegen eines neuen Buches ausgefüllt sind.}
	\fr{Das System muss sicherstellen, dass Administratoren das Antwortfeld ausfüllen, bevor sie eine Antwort absenden können.}
	\fr{Das System muss im Ausleihverlauf eines Nutzers sowohl das Ausleihdatum als auch das Rückgabedatum jedes Buches anzeigen.}
	\fr{Bei verspäteter Rückgabe von Büchern soll das System die fälligen Gebühren automatisch berechnen und über die Stripe-API zur Zahlung auffordern.}
	\fr{Für jede Buchbeschreibung muss ein Button vorhanden sein, mit dem der Nutzer die Beschreibung zwischen Deutsch und Englisch umschalten kann.}
	\fr{Wenn ein Buch gelöscht wurde, darf der Benutzer die Ausleihe dieses Buches nicht mehr verlängern können.}
	\fr{Wenn ein Buch gelöscht wird, muss der Benutzer in der Ausleihliste darüber informiert werden, dass dieses Buch gelöscht wird.}
	

	
	\caption{Funktionale Anforderungen von LibraNova}
	\label{tab:functional-requirements}
\end{longtable}


\begin{longtable}{|c|p{13cm}|}
	\hline
	\textbf{ID} & \textbf{Anforderung (Beschreibung)} \\
	\hline
	\endfirsthead
	
	\hline
	\textbf{ID} & \textbf{Anforderung (Beschreibung)} \\
	\hline
	\endhead
	
	\nfr{Das System muss eine responsive Benutzeroberfläche bieten, die auf verschiedenen Geräten und Bildschirmgrößen funktioniert.}
	\nfr{Die Anwendung muss eine sichere Kommunikation über HTTPS (TLS) gewährleisten, um die Datenübertragung zwischen Frontend und Backend zu schützen (Ports 8443 und 3000).}
	\nfr{Das System muss sicherstellen, dass Benutzer nur mit gültigen und nicht abgelaufenen Tokens Zugriff auf geschützte Ressourcen erhalten.}
	\nfr{Die Reaktionszeit der Such- und Filterfunktionen im Frontend darf 2 Sekunden nicht überschreiten, um ein flüssiges Nutzererlebnis zu gewährleisten.}
	\nfr{Die Anwendung muss grundlegende Barrierefreiheitsanforderungen gemäß WCAG 2.1 (Level AA) erfüllen, um auch Nutzern mit Einschränkungen einen uneingeschränkten Zugang und eine barrierefreie Nutzung zu ermöglichen.}
	\nfr{Die REST-APIs müssen robust gegen fehlerhafte Eingaben sein und validierte Anfragen verarbeiten, um die Stabilität des Systems zu gewährleisten.}
	\nfr{Die Backend-Services sollen modular aufgebaut sein, um einfache Wartbarkeit und Erweiterbarkeit zu ermöglichen.}
	\nfr{Die API-Dokumentation muss aktuell und entwicklerfreundlich sein.}
	\nfr{Die Anwendung muss eine klare Trennung zwischen Benutzerrollen (Nutzer, Administrator) ermöglichen.}
	\nfr{Die Anwendung muss auf allen gängigen Webbrowsern wie Firefox, Chrome und Safari ohne Einschränkungen lauffähig sein.}
	\nfr{Der Anmeldevorgang eines Nutzers (ab dem Klick auf den Login-Button bis zur erfolgreichen Authentifizierung und Weiterleitung zur Startseite) darf nicht länger als 3 Sekunden dauern.}
	\nfr{Die Benutzeroberfläche muss auf allen Seiten konsistente Navigationselemente, Schaltflächen-Designs und Layouts verwenden, um eine einheitliche Nutzererfahrung sicherzustellen.}
	\nfr{Die Benutzerführung muss intuitiv gestaltet sein, sodass typische Nutzeraktionen wie Buchsuche, Ausleihe oder Bewertung ohne zusätzliche Anleitung verständlich und ausführbar sind.}
	\nfr{Das System muss sicherstellen, dass nur autorisierte Clients auf geschützte Endpunkte wie \texttt{/api/books/secure/**} oder \texttt{/api/admin/secure/**} zugreifen können.}
	\nfr{	Die Anwendung muss vollständig auf Deutsch und Englisch verfügbar sein, wobei alle Seiten und Benutzeroberflächen entsprechend übersetzt werden.}
	

	\caption{Nicht-funktionale Anforderungen von LibraNova}
	\label{tab:non-functional-requirements}
\end{longtable}



\begin{longtable}{|c|p{13cm}|}
	\hline
	\textbf{ID} & \textbf{Anforderung (Beschreibung)} \\
	\hline
	\endfirsthead
	
	\hline
	\textbf{ID} & \textbf{Anforderung (Beschreibung)} \\
	\hline
	\endhead
	
	\tr{Das Backend muss HTTPS-Verbindungen über Port 8443 ermöglichen, mithilfe eines selbstsignierten SSL-Zertifikats.}
	\tr{Der Zugriff auf geschützte API-Endpunkte muss über JWT-Token abgesichert werden (z., B.\ /api/books/secure/**).}
	\tr{CSRF-Schutz muss für REST-APIs deaktiviert werden, da die Authentifizierung stateless über JWT erfolgt.}
	\tr{Das Backend muss eine RESTful API zur Verfügung stellen, um Daten zwischen Frontend und Backend zu übertragen.}
	\tr{Die API-Endpunkte müssen CORS-konform konfiguriert sein, sodass ausschließlich Anfragen von der vertrauenswürdigen Frontend-Domain \texttt{https://localhost:3000} zugelassen werden.}
	\tr{Die Authentifizierung und Autorisierung erfolgt über Okta als Identity Provider unter Verwendung von OAuth2, OpenID Connect und JWT, um sichere und standardisierte Benutzerzugriffe zu gewährleisten.}
	\tr{Die REST-API muss mit Swagger dokumentiert sein und über Swagger UI erreichbar sein.}
	\tr{Die Anwendung muss HTTP-Methoden wie POST, PUT, DELETE und PATCH für bestimmte Ressourcen (z.\,B.\ \texttt{Book}, \texttt{Review}, \texttt{Message}) deaktivieren, wenn sie nicht benötigt werden.}
	\tr{Die Anwendung muss eine relationale MySQL-Datenbank verwenden (libranova\_db), angebunden über JDBC.}
	\tr{Die Anwendung muss Spring Data JPA zur Datenpersistenz verwenden.}
	\tr{Die Anwendung muss Entities wie Book, Review und Message als JPA-Entitäten modellieren und mittels Repository-Schnittstellen verfügbar machen.}
	\tr{Die Anwendung muss Spring Data REST verwenden, um Repository-Schnittstellen automatisch als RESTful API verfügbar zu machen.}
	\tr{Lombok wird zur Reduzierung von Boilerplate-Code in Entities und anderen Klassen verwendet.}
	\tr{Die Anwendung muss Spring Boot Web Starter verwenden, um Web-Server-Funktionalitäten und REST-Controller bereitzustellen.}
	\tr{Die Anwendung muss Unit-Tests mit JUnit und Mocking mit Mockito implementieren, um die Funktionalität des Backends abzusichern.}
	\tr{Das Frontend muss in React umgesetzt sein und via HTTPS über Port 3000 laufen.}
	\tr{Das Frontend muss mit dem Backend über REST-APIs kommunizieren.}
	\tr{Das Frontend muss die vom Backend bereitgestellten Endpunkte nutzen und Authentifizierung über Tokens abwickeln.}
	\tr{Die Internationalisierung der Anwendung wird im Frontend mit React unter Verwendung der Bibliothek i18next umgesetzt.}
	

	
	
	\caption{Technische Anforderungen von LibraNova}
	\label{tab:technical-requirements}
\end{longtable}