\chapter{Methodologie und Systementwurf }



\section{Front-End Technologien}\index{Front-End Technologien}


\section{Back-End Technologien}\index{Back-End Technologien}


\subsubsection{Absicherung der REST-Endpunkte mit Spring Security}

Zur Absicherung der REST-Endpunkte in „LibraNova“ wurde das Framework \textbf{Spring Security} in Kombination mit OAuth2 und JSON Web Tokens (JWT) eingesetzt. Dabei schützt eine zentrale Sicherheitskonfiguration gezielt sensible Pfade wie \texttt{/api/books/secure/\***}, Alle übrigen Endpunkte bleiben öffentlich zugänglich. Diese Trennung zwischen geschützten und öffentlichen Routen erlaubt eine kontrollierte Zugriffskontrolle und gewährleistet gleichzeitig Offenheit für nicht sensible Daten. \\ 
Die Konfiguration erfolgt über eine \texttt{SecurityFilterChain}-Bean. Hierbei wurde die CSRF-Absicherung deaktiviert, da sie bei stateless JWT-Authentifizierung überflüssig ist. Gleichzeitig wird eine OAuth2-Resource-Server-Konfiguration mit JWT-Validierung verwendet. Die Einrichtung einer \texttt{ContentNegotiationStrategy} unterstützt eine saubere Inhaltsaushandlung zwischen Client und Server. Die Integration der \texttt{Okta.configureResourceServer401ResponseBody(http)}-Methode verbessert zudem die Fehlerbehandlung bei unautorisierten Zugriffen.

\begin{lstlisting}[language=Java, caption={Spring Security-Konfiguration}]
	http
	.csrf(csrf -> csrf.disable())
	.authorizeHttpRequests(configurer -> configurer
	.requestMatchers("/api/books/secure/**", 
	"/api/reviews/secure/**", 
	"/api/messages/secure/**", 
	"/api/admin/secure/**").authenticated()
	.anyRequest().permitAll())
	.oauth2ResourceServer(oauth2 -> oauth2.jwt())
	.cors(cors -> cors.configurationSource(corsConfigurationSource()));
\end{lstlisting}
Durch diese Konfiguration wird sichergestellt, dass nur authentifizierte Nutzer mit gültigem JWT-Token auf geschützte Ressourcen zugreifen können, während gleichzeitig der Zugriff auf öffentliche Inhalte uneingeschränkt möglich bleibt.

\subsubsection{CORS-Konfiguration und Einschränkung von HTTP-Methoden}

Zur zusätzlichen Absicherung des Backends wurden zwei Maßnahmen umgesetzt: Erstens wurde eine \textbf{CORS-Konfiguration} implementiert, die ausschließlich Anfragen vom React-Frontend (\texttt{https://localhost:3000}) erlaubt. Zweitens wurden mithilfe von \texttt{RepositoryRestConfigurer} bestimmte HTTP-Methoden wie \texttt{POST}, \texttt{PUT}, \texttt{PATCH} und \texttt{DELETE} auf ausgewählten Entitäten deaktiviert, um ungewollte Änderungen über Spring Data REST zu verhindern.

\begin{lstlisting}[language=Java, caption={CORS und HTTP-Methodenbeschränkung}]
	config.exposeIdsFor(Book.class, Review.class, Message.class);
	
	HttpMethod[] unsupported = {POST, PUT, PATCH, DELETE};
	restrictHttpMethods(Book.class, config, unsupported);
	
	corsRegistry.addMapping(config.getBasePath() + "/**")
	.allowedOrigins("https://localhost:3000");
\end{lstlisting}
Diese Konfiguration trägt maßgeblich zur Sicherheit und Stabilität der Anwendung bei, indem sie sowohl die erlaubten Ursprünge als auch die zulässigen Zugriffsarten explizit definiert.



\subsubsection{SSL-Konfiguration im Backend}

Zur Absicherung des Datenverkehrs wurde in der Anwendung eine HTTPS-Konfiguration implementiert, bei der der Server auf Port \texttt{8443} HTTPS-Anfragen entgegennimmt (siehe \ref{HTTPS-Config}). Dafür wurde SSL/TLS aktiviert, um eine verschlüsselte Kommunikation zwischen Client und Server zu gewährleisten. \\
Die Verschlüsselung basiert auf einem selbstsignierten SSL-Zertifikat, das mit dem folgenden Befehl generiert wurde:
\begin{lstlisting}[language=bash, caption={Generierung eines SSL-Zertifikats}, breaklines=true]
	keytool -genkeypair -alias libranova \
	-keystore src/main/resources/libranova-keystore.p12 \
	-keypass secret -storepass secret -storeType PKCS12 \
	-keyalg RSA -keysize 2048 -validity 365 \
	-dname "C=DE, ST=Rhineland-Palatinate, L=Trier, O=libranova, OU=Studies Backend, CN=localhost" \
	-ext "SAN=dns:localhost"
\end{lstlisting}
Dabei wurde eine Keystore-Datei im \texttt{PKCS12}-Format (\texttt{libranova-keystore.p12}) erstellt, in der das Zertifikat unter dem Alias \texttt{libranova} gespeichert ist. Diese Datei wird in der \texttt{application.properties} wie folgt eingebunden:

\begin{lstlisting}[language=, label=HTTPS-Config, caption={HTTPS- und SSL-Konfiguration}]
	# HTTPS settings
	server.port=8443
	server.ssl.enabled=true
	server.ssl.key-alias=libranova
	server.ssl.key-store=classpath:libranova-keystore.p12
	server.ssl.key-store-password=secret
	server.ssl.key-store-type=PKCS12
\end{lstlisting}
Durch diese Konfiguration wird sichergestellt, dass alle übermittelten Daten verschlüsselt und vor unbefugtem Zugriff geschützt sind. Die Anwendung erfüllt damit grundlegende Sicherheitsanforderungen moderner Webanwendungen.


\subsubsection{SSL-Konfiguration im Frontend}

Auch im Frontend wurde eine lokale HTTPS-Verbindung eingerichtet, um eine verschlüsselte Kommunikation mit dem Backend zu ermöglichen. Dazu wurde mithilfe von OpenSSL ein selbstsigniertes Zertifikat erstellt:

\begin{lstlisting}[language=bash, caption={Generierung eines Frontend-Zertifikats}]
	openssl req -x509 \
	-out ssl-localhost-libranova/localhost.crt \
	-keyout ssl-localhost-libranova/localhost.key \
	-newkey rsa:2048 -nodes -sha256 -days 365 \
	-config localhost.conf
\end{lstlisting}
Dieses Kommando erzeugte zwei Dateien:
\begin{itemize}
	\item \texttt{localhost.crt} – das Zertifikat
	\item \texttt{localhost.key} – der zugehörige private Schlüssel
\end{itemize}
In der \texttt{.env}-Datei wurden diese Dateien anschließend referenziert, um die React-Entwicklungsumgebung über HTTPS zu betreiben:

\begin{lstlisting}[language=bash, caption={.env Konfiguration für HTTPS und API-Zugriff}]
	SSL_CRT_FILE=ssl-localhost-libranova/localhost.crt
	SSL_KEY_FILE=ssl-localhost-libranova/localhost.key
	REACT_APP_API_URL='https://localhost:8443/api'
\end{lstlisting}
Die Konfigurationsdatei \texttt{localhost.conf} enthielt die benötigten Informationen für das Zertifikat (wie Standort und Common Name), um den Generierungsprozess zu automatisieren:

\begin{lstlisting}[caption={localhost.conf}]
	[req]
	prompt = no
	distinguished_name = dn
	
	[dn]
	C = DE
	ST = Rhineland-Palatinate
	L = Trier
	O = Libranova
	OU = Studies
	CN = localhost
\end{lstlisting}
Diese Konfiguration ermöglicht eine sichere Verbindung zwischen Frontend und Backend während der lokalen Entwicklung. \\ \\
Die HTTPS-Implementierung mit SSL/TLS gewährleistet die Einhaltung von Sicherheitsstandards und schützt sensible Nutzerdaten zuverlässig.
















