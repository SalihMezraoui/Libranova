\cleardoublepage

\chapter*{Zusammenfassung}
\sloppy
\justifying

Diese Arbeit beschäftigt sich mit der Konzeption und Umsetzung von \textit{LibraNova}, einer modernen, webbasierten Full-Stack-Anwendung zur digitalen Verwaltung einer Bibliothek sowie zur effizienten Organisation von Buchausleihen. Ziel des Projekts ist es, sowohl Nutzer:innen als auch Administrator:innen eine intuitive, leistungsfähige und sichere Plattform bereitzustellen, die den gesamten Ausleihprozess digital unterstützt.  

\noindent Die Anwendung wurde mit \textbf{Spring Boot} im Backend und \textbf{React} mit TypeScript im Frontend realisiert. Nutzer:innen können Bücher recherchieren, deren Verfügbarkeit in Echtzeit prüfen sowie Rezensionen lesen und verfassen. Administrator:innen erhalten Werkzeuge zur Verwaltung des Buchbestands und zur Kommunikation mit den Nutzer:innen.  

\noindent Zur Absicherung sensibler Funktionen werden moderne Authentifizierungs- und Autorisierungsverfahren eingesetzt, darunter \textbf{JWT} und \textbf{OAuth2} via Okta. Für kostenpflichtige Bibliotheksdienste ist die \textbf{Stripe API} eingebunden, um sichere Zahlungen zu ermöglichen. Die persistente Datenhaltung erfolgt über die \textbf{MySQL-Datenbank}. 

\noindent Die Anwendung ist in \textbf{deutscher und englischer Sprache} verfügbar und legt großen Wert auf \textbf{Barrierefreiheit}, um einen breiten Nutzerzugang zu gewährleisten. Im Rahmen der Entwicklung wurde besonderer Wert auf eine responsive Benutzeroberfläche, ein konsistentes UX-Design sowie die Qualitätssicherung durch automatisierte Tests mit \textbf{JUnit} und \textbf{Mockito} gelegt. \textit{LibraNova} demonstriert, wie durch den Einsatz moderner Webtechnologien eine skalierbare und benutzerfreundliche Lösung für das Bibliotheksmanagement realisiert werden kann.