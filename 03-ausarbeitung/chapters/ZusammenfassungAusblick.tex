\chapter{Zusammenfassung und Ausblick}

Diese Dokumentation hat die Konzeption, Implementierung und Analyse der Full-Stack-Bibliotheksmanagement-Anwendung „LibraNova“ umfassend dargestellt. Ziel des Projekts war es, eine moderne, benutzerfreundliche und sichere Plattform zu entwickeln, die den Anforderungen der Bibliotheksnutzer und Administratoren gerecht wird. Mit bewährten Technologien wie Spring Boot im Backend, React mit TypeScript im Frontend sowie einer Sicherheitsarchitektur mit Okta, JWT, OAuth2 und OpenID Connect konnte eine skalierbare und wartbare Lösung realisiert werden.

Die klare Trennung von Backend und Frontend sowie moderne Frameworks ermöglichten eine effiziente Entwicklung. Spring Security in Verbindung mit Okta sorgt für einen robusten Authentifizierungs- und Autorisierungsmechanismus, der die Sicherheit sensibler Daten gewährleistet. Die Integration von Redux im Frontend unterstützt ein effektives State-Management für eine reaktive und performante Benutzeroberfläche.

Die Anwendung umfasst Kernfunktionen zur Buchsuche, Ausleihe und Bewertung sowie administrative Werkzeuge zur Verwaltung des Buchkatalogs und der Nutzeranfragen. Die Qualitätssicherung erfolgte mit JUnit, Mockito und React Testing Library, um Backend- und Frontend-Komponenten systematisch zu testen und die Stabilität der Anwendung sicherzustellen. Postman diente als praktisches API-Testtool und ergänzte den Entwicklungsprozess.

Zukünftige Arbeiten könnten Cloud-Bereitstellung mit CI/CD-Pipelines, erweiterte Testabdeckung sowie Funktionen wie Benachrichtigungen oder personalisierte Empfehlungen umfassen. Die modulare Architektur erlaubt eine flexible Anpassung von LibraNova an unterschiedliche Bibliotheksanforderungen.

Insgesamt bietet LibraNova eine zeitgemäße Lösung, die den digitalen Wandel im Bibliotheksmanagement unterstützt und die Nutzererfahrung durch intuitive Bedienung und sichere Prozesse verbessert. Diese Dokumentation bildet eine Grundlage für weitere Entwicklungen und zeigt, wie moderne Webtechnologien effektiv kombiniert werden können, um praxisnahe und nachhaltige Anwendungen zu schaffen.