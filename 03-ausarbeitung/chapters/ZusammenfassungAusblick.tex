\chapter{Resümee und Ausblick}

In dieser Arbeit wurde die Konzeption und Realisierung einer modernen, benutzerfreundlichen und sicheren Bibliotheksmanagement-Plattform namens LibraNova vorgestellt. Das Ziel war es, eine skalierbare und wartbare Lösung zu entwickeln, die sowohl den Anforderungen der Nutzerinnen und Nutzer als auch der Bibliotheksadministration gerecht wird. Dabei stand die Integration aktueller Webtechnologien wie Spring Boot im Backend, React mit TypeScript im Frontend sowie eine umfassende Sicherheitsarchitektur mit Okta, JWT, OAuth2 und OpenID Connect im Fokus. Zudem wurde die Plattform um eine Zahlungsfunktion erweitert, die die Stripe API nutzt, um beispielsweise etwaige Gebühren reibungslos und sicher abzuwickeln.

\noindent Die Anwendung ist mehrsprachig ausgelegt und unterstützt sowohl Deutsch als auch Englisch, um eine breitere Nutzerbasis anzusprechen. Darüber hinaus legt das System großen Wert auf Responsiveness, sodass die Plattform auf verschiedenen Endgeräten — sei es Desktop, Tablet oder Smartphone — optimal genutzt werden kann. Die Umsetzung erfolgte durch modulare Architekturen, die REST-APIs nutzen, um eine flexible Anpassung an verschiedene Nutzungsszenarien zu ermöglichen. Wesentliche Funktionen wie die Buchsuche, das Ausleihmanagement, Nutzerbewertungen und die administrative Verwaltung wurden effizient implementiert und durch systematische Tests abgesichert.

\noindent Die Plattform unterstützt zudem zukünftige Erweiterungen, etwa Cloud-Bereitstellung, erweiterte Testabdeckung sowie personalisierte Empfehlungen und Benachrichtigungen. Mit LibraNova wurde eine innovative Lösung geschaffen, die den digitalen Wandel im Bibliothekswesen aktiv mitgestaltet und die Nutzererfahrung durch intuitive Bedienung, hohe Sicherheitsstandards sowie eine responsive Gestaltung deutlich verbessert. Das Projekt bildet eine solide Grundlage für zukünftige Entwicklungen und bietet zahlreiche Potenziale zur weiteren Optimierung und Erweiterung.